\documentclass[
	xcolor=dvipsnames,
	10pt, 
	]{beamer}
\graphicspath{{./fig/}}
\RequirePackage{"./sty/UserPackageManager"}

%% TUBS
  \usebackgroundtemplate%
{
	\includegraphics[width=\paperwidth,height=\paperheight]{./bonos/Bonobono}%
}


\title[]{Notes on Baryon Polarization}
\author[\frac{1}{2}uthor]{Bono}
\institute{Bono University}
\date{\today}
\subtitle{\today}


\begin{document}


%%%%%%%%%%%%%%% Title page 
\begin{frame}[t,plain] % Cover slide
       \titlepage
\end{frame}
\begin{frame}{Spin State Representation of Baryon Decay}
	\begin{block}{}
		Consider a decay process where spin-$\frac{1}{2}$ baryon($\Xi^-$)  decays into another spin-$\frac{1}{2}$ baryon ($\Lambda$) and a spin-0 meson($\pi^-$). In quantum mechanics, this decay $\Xi^-\to \Lambda\pi^-$  can be modeled as a transition from the initial state $\ket{\Xi^-}$ to the final state $\ket{\Lambda,\pi^-}$. Since $\pi^-$ is a spin-0 meson, we focues on the orbital angular momentum carried by $\pi^-$. In the S-wave transition  $\pi^-$ carries no angular momentum($l=0$), while in the P-wave transition it carries an angular momentum of $l=1$.  The initial state of $\Xi^-$ can be defined as		$\Xi^-=\ket{\frac{1}{2},\frac{1}{2}}$. The decay process is then represented as:
		\begin{align}
			\ket{\frac{1}{2},\frac{1}{2}}_\Xi &= A \ket{\frac{1}{2},\frac{1}{2}}_\Lambda\ket{0,0}_\pi Y_0^0 +B[C_{\frac{1}{2},\frac{1}{2},1,0}^{\frac{1}{2},\frac{1}{2}}\ket{\frac{1}{2},\frac{1}{2}}_\Lambda\ket{1,0}_\pi Y_1^0\nonumber\\
			&+C_{\frac{1}{2},-\frac{1}{2},1,1}^{\frac{1}{2},\frac{1}{2}}\ket{\frac{1}{2},-\frac{1}{2}}_\Lambda\ket{1,1}\pi Y_1^1]\label{Decay}
		\end{align}
	\end{block}
\end{frame}
\begin{frame}{Spin State Representation of Baryon Decay}
	\begin{block}{}
	Here A and B represent the transition amplitudes of S-wave and P-wave, respectively. The terms $C_{j1,m1,j2,m2}^{J,M}$  are Clebsch-Gordan coefficients, and $Y_l^m(\theta,\phi)$s denote spherical harmonics. Note that A and B are complex numbers. Focusing on the spin state of $\Lambda$, we rewrite \eqref{Decay} as:
	\begin{align}
		\ket\Lambda = (a + b\cos\theta)\ket{\frac{1}{2},\frac{1}{2}}_\Lambda + be^{i\phi}\sin\theta \ket{\frac{1}{2},-\frac{1}{2}}_\Lambda.\label{State}
	\end{align}
	The decay spectrum of $\Lambda$ can be expressed as:
	\begin{align}
		\bra\Lambda\ket\Lambda = (a+b\cos\theta)^* (a+b\cos\theta)\Lambda_{\uparrow}+ (be^{i\phi})^*be^{i\phi}\Lambda_{\downarrow}\label{Lambda}
	\end{align}
	Here, $\Lambda_{\uparrow}$ and $\Lambda_{\downarrow}$ represent the spin-up and spin-down states of $\Lambda$, respectively.
\end{block}
\end{frame}
\begin{frame}{Baryon Decay Parameters}
	\begin{block}{}
		If we not distinguish $\Lambda_{\uparrow}$ and $\Lambda_{\downarrow}$, \eqref{Lambda} turns into:
		\begin{align}
			\Lambda(\theta,\phi)=\abs{a}^2+\abs{b}^2+2\re[a^*b]\cos\theta = 1+\alpha\cos\theta.\label{Spectra}
		\end{align}
		The wavefunction should be normalized at the final calculation, but at the moment we will require $\abs{a}^2+\abs{b}^2=1$ and introduce decay parameters:
		\begin{align}
			\alpha= \re[a^*b], \beta = \im[a^*b],\gamma = \abs{a}^2-\abs{b}^2.
		\end{align}
		Remind that our coordinate system here is the Center-of-Mass frame of $\Xi^-$, with spin pointing the z axis as we already set the initial state as $\ket\Xi=\ket{\frac{1}{2},\frac{1}{2}}$. $\theta$ and $\phi$ corresponds to the spherical variables of $\Lambda$ momentum at CoM of $\Xi^-$.
	\end{block}
\end{frame}

\begin{frame}{Spin State of Daughter Baryon}
	\begin{block}{}
		Now time has come to calculate the spin state of $\Lambda$ in \textbf{CoM of }$\mathbf{\Xi^-}$. The spinnor representation of $\Lambda$, as given in \eqref{State}, is:
		\begin{align}
			\ket\Lambda = \begin{pmatrix}
		 a + b\cos\theta\\
		  be^{i\phi}\sin\theta 
			\end{pmatrix}\label{Spinnor}
		\end{align}
		and the direction of spin, or, the \textit{polarization} of $\Lambda$ is
		\begin{align}
			\vec P=\bra\Lambda \hat S\ket \Lambda.
		\end{align}
		Here we defined the spin-operator $\hat S = \{\sigma_x,\sigma_y,\sigma_z\}$, where $\sigma_i$s are the \textit{Pauli matrices}.
		Our task is to calculate each component of $\vec P$, \textit{at Center of Mass of $\Lambda$}.
	\end{block}
\end{frame}
\begin{frame}{Calculating $P_x$}
	\begin{block}{}
		We start with calculating $P_x$. $\sigma_x = \dmat 0 1 1 0$ and
		\begin{align}
			P_x& =\bra\Lambda \dmat 0 1 1 0 \ket \Lambda\nonumber\\
			& = (a +b \cos\theta)^*be^{i\phi}\sin\theta+(be^{i\phi}\sin\theta)^*(a +b \cos\theta)\nonumber\\
			&= (a^*be^{i\phi}+ab^*e^{-i\phi})\sin\theta + \abs{b}^2\sin\theta\cos\theta(e^{i\phi}+e^{-i\phi})\nonumber\\
			&=(2\re[a^*b]\cos\phi-2 \im[a^*b]\sin\phi )\sin\theta+2\abs{b}^2\sin\theta\cos\theta\cos\phi\nonumber\\
			&= [\alpha\cos\phi-\beta\sin\phi+ (1-\gamma)\cos\theta\cos\phi]\sin\theta.
		\end{align}
		Here we used $2\abs{b}^2 = \abs{a}^2+\abs{b}^2 - (\abs{a}^2-\abs{b}^2)=1-\gamma$.
	\end{block}
\end{frame}
\begin{frame}{Calculating $P_y$}
	\begin{block}{}
		\begin{align}
	P_y& =\bra\Lambda \dmat {0}{-i}{i}{0} \ket \Lambda\nonumber\\
	& =-i (a +b \cos\theta)^*be^{i\phi}\sin\theta+i(be^{i\phi}\sin\theta)^*(a +b \cos\theta)\nonumber\\
	&= -i[(a^*be^{i\phi}-ab^*e^{-i\phi})\sin\theta + \abs{b}^2\sin\theta\cos\theta(e^{i\phi}-e^{-i\phi})]\nonumber\\
	&=-i[(2i\im[a^*b]\cos\phi+2 i\re[a^*b]\sin\phi )\sin\theta+2i\abs{b}^2\sin\theta\cos\theta\sin\phi]\nonumber\\
	&= [\beta\cos\phi+\alpha\sin\phi+ (1-\gamma)\cos\theta\sin\phi]\sin\theta.
\end{align}
	\end{block}
\end{frame}
\begin{frame}{Calculating $P_z$}
	\begin{block}{}
		\begin{align}
			P_z& =\bra\Lambda \dmat {1}{0}{0}{-1} \ket \Lambda\nonumber\\
			&= (a+b\cos\theta)^*(a+b\cos\theta)-(b^*e^{-i\phi})be^{i\phi}\nonumber\\
			&= \abs {a}^2 + (a^*b+ab^*)\cos\theta + \abs{b}^2\cos^2\theta -\abs{b}^2\sin^2\theta\nonumber\\
			&= \frac{1}{2}(1+\gamma) + 2\re[a^*b]\cos\theta  + \frac{\cos^2\theta-\sin^2\theta}{2}(1-\gamma)\nonumber\\
			&=\frac{1}{2}(1+\gamma) + \alpha\cos\theta  + \frac{\cos^2\theta-\sin^2\theta}{2}(1-\gamma)
.		\end{align}
	\end{block}
\end{frame}
\begin{frame}{CoM Frame of $\Lambda$ }
	\begin{block}{}
		Untill now we calculated $\Lambda$ polarization vector in CoM of $\Xi$. In order to obtain meaningful results, we need to represent  it in $\Lambda$ CoM. We would define the orientation of $\Lambda$ CoM frame as:
		\begin{align}
			\begin{cases}
				\hat z_\Lambda = \hat \Lambda=(\sin\theta\cos\phi,\sin\theta\sin\phi,\cos\theta)\\
				\hat x_\Lambda = \frac{\hat P_\Xi\times\hat \Lambda}{\abs{\hat P_\Xi\times\hat \Lambda}}=(-\sin\phi,\cos\phi,0)\\
				\hat y_\Lambda= \hat z_\Lambda\times\hat x_\Lambda=(-\cos\theta\cos\phi,-\cos\theta\sin\phi,\sin\theta)
			\end{cases}\label{LCoord}
		\end{align}
		and obtain the representation of $\vec P$ as:
		\begin{align}
			\begin{cases}
				P_{x,\Lambda} = \vec P \cdot \hat x_\Lambda\\
				P_{y,\Lambda} = \vec P \cdot \hat y_\Lambda\\
				P_{z,\Lambda} = \vec P \cdot \hat z_\Lambda
			\end{cases}
		\end{align}
	\end{block}
\end{frame}
\begin{frame}{$P_{x,\Lambda}$}
	\begin{block}{}
		We should calculate each component of $\vec{P}_\Lambda$. Remind that, we stated by setting polarization of $\Xi$,$\vec P_{\Xi}=(0,0,1)$. However, $\Lambda$ CoM is defined relative to  $\vec P_{\Xi}$ and $\hat\Lambda$. Then, without loss of generality, we represent the polarization of $\Lambda$ in terms of $\Xi$ polarization and $\Lambda$ momentum direction.
		\begin{align}
			P_{x,\Lambda} =& -\sin\phi[\alpha\cos\phi-\beta\sin\phi+ (1-\gamma)\cos\theta\cos\phi]\sin\theta\nonumber\\
			&+\cos\phi[\beta\cos\phi+\alpha\sin\phi+ (1-\gamma)\cos\theta\sin\phi]\sin\theta\nonumber\\
			&=\beta\sin\theta = \beta (\vec{P}_{\Xi}\times \hat\Lambda)\cdot \hat x\label{PLx}
		\end{align}
		Here we used the fact that, From \eqref{LCoord}, $\vec{P}_{\Xi}\times \hat\Lambda=\abs{\vec P_{\Xi}\times\hat\Lambda}\hat x=\sin\theta\hat x$.
	\end{block}
\end{frame}
\begin{frame}{$P_{y,\Lambda}$}
	\begin{block}{}
		\begin{align}
				P_{y,\Lambda} =&-\cos\theta\cos\phi[\alpha\cos\phi-\beta\sin\phi+ (1-\gamma)\cos\theta\cos\phi]\sin\theta.\nonumber\\
				&-\cos\theta\sin\phi[\beta\cos\phi+\alpha\sin\phi+ (1-\gamma)\cos\theta\sin\phi]\sin\theta.
				\nonumber\\
				&+\sin\theta[\frac{1}{2}(1+\gamma) + \alpha\cos\theta  + \frac{\cos^2\theta-\sin^2\theta}{2}(1-\gamma)]\nonumber\\
				=&\sin\theta[\frac{1}{2}(1+\gamma)+(1-\gamma)(\frac{\cos^2\theta-\sin^2\theta}{2}-\cos^2\theta)]\nonumber\\
				=&\sin\theta\gamma = \gamma \hat \Lambda\times(\vec{P}_\Xi\times\hat\Lambda)\cdot \hat y\label{PLy}
		\end{align}
	\end{block}
\end{frame}
\begin{frame}{$P_{z,\Lambda}$}
	\begin{block}{}
		\begin{align}
			P_{z,\Lambda} =&\sin\theta\cos\phi [\alpha\cos\phi-\beta\sin\phi+ (1-\gamma)\cos\theta\cos\phi]\sin\theta\nonumber\\
			+&\sin\theta\sin\phi[\beta\cos\phi+\alpha\sin\phi+ (1-\gamma)\cos\theta\sin\phi]\sin\theta\nonumber\\
			+&\cos\theta[\frac{1}{2}(1+\gamma) + \alpha\cos\theta  + \frac{\cos^2\theta-\sin^2\theta}{2}(1-\gamma)]\nonumber\\
			=&\alpha+\cos\theta[\frac{1+\gamma}{2}+(1-\gamma)(\frac{\cos^2\theta-\sin^2\theta}{2}+\sin^2\theta)]\nonumber\\
			=&\alpha+\cos\theta\label{PLz}
		\end{align}
	\end{block}
\end{frame}
\begin{frame}{Summary}
	\begin{block}{}
		To summarize \eqref{PLx} - \eqref{PLz}, we have
		\begin{align*}
			\vec{P}_\Lambda \propto \beta \vec{P_\Xi}\times\hat \Lambda + \gamma\hat \Lambda\times( \vec{P_\Xi}\times\hat \Lambda)+(\alpha+\cos\theta)\hat\Lambda
		\end{align*}
		However, we should note that $\abs{\vec P}_\Lambda=1$. We need a normalization. Referring to  \eqref{Spectra} normalization parameter seems to be obvious, but we will show the derivation here. Remind that $\alpha^2+\beta^2+\gamma^2=1$.
		\begin{align}
			N&=\sqrt{(\alpha+\cos\theta)^2+(\beta\sin\theta)^2+(\gamma\sin\theta)^2}\nonumber\\
			&=\sqrt{\alpha^2+2\alpha\cos\theta +\cos^2\theta + \cancelto{1-\alpha^2}{(\beta^2+\gamma^2)}(1-\cos^2\theta)}\nonumber\\
			&= \sqrt{1 +2\alpha\cos\theta+\alpha^2\cos^2\theta}=1+\alpha\cos\theta
		\end{align}
	\end{block}
\end{frame}
\begin{frame}{Summary}
	\begin{block}{}
		Now we obtained normalization parameter. Then we have
		\begin{align}
			\vec{P}_\Lambda=\frac{(\alpha+\cos\theta)\hat\Lambda+\beta(\vec P_\Xi\times\hat\Lambda)+\gamma\hat\Lambda\times(\vec P_\Xi\times\hat\Lambda)}{1+\alpha\cos\theta}
		\end{align}
	\end{block}
\end{frame}
\end{document}
